\documentclass[UTF8]{ctexart}
\usepackage{graphicx}
\usepackage{float}
\usepackage{geometry}
\usepackage{fancyhdr}
\usepackage{lastpage}
\usepackage{enumerate}
\geometry{left=2.54cm,right=2.54cm,top=3.18cm,bottom=3.18cm}%页边距
\pagestyle{fancy}
\lhead{\includegraphics[scale=1]{sjtu-logo-red.pdf}}  
\rhead{BI908 多种方法实现的脑肿瘤分割} 
\cfoot{第 \thepage\ 页\ 共 \pageref{LastPage} 页} 

\begin{document}

\begin{titlepage}
    \begin{center}
        \includegraphics[width=0.8\textwidth]{sjtu-name-blue.pdf}\\[1cm]
        \textsc{\Huge \bfseries 课程报告}\\[1.5cm]
        \includegraphics[width=0.3\textwidth]{sjtu-badge-blue.pdf}\\[0.5cm]    

        \Huge \bfseries{BI908 脑肿瘤分割项目报告}\\[1cm]
        \Large \bfseries{518021910971 裴奕博}\\
        \Large \bfseries{学号 丁一}\\
        \Large \bfseries{学号 陈波}\\
        \Large \bfseries{学号 栗行健}
    \end{center}
\end{titlepage}
\tableofcontents
%正文

\newpage
\section{项目简介与预处理}
\subsection{项目简介}
本项目采用了阈值分割,区域增长,深度学习等多种分割算法,结合锐化滤波,形态学处理等辅助增强手段,对给定的脑肿瘤进行了分割,并取得了不错的效果。整个项目均采用自己实现的Python算法,项目的总流程如下:
\begin{figure}[H]
    \centering  %图片全局居中
    \includegraphics[width=\textwidth]{figure/workflow.png}
    \caption{总工作流程图}
\end{figure}
其中:
% TODO
\begin{enumerate}[1)]
    \item nii.gz文件的输入输出均由SimpleITK包完成
    \item 数据预处理部分的算法包括 % TODO
    \item 图像分割算法包括:去除背景的三维Otsu算法,传统的三维区域增长算法,改进后的三维区域增长算法。
    \item 分割结果的形态学后处理方法包括:开运算和闭运算操作。
\end{enumerate}

我们组的编号为04,采用的数据集是Dataset\_Group/04文件夹下的三个待分割样本,编号分别为$BRAT\_008,BRAT\_033,BRAT\_259$。


\subsection{各文件(夹)功能}
\begin{itemize}
    \item README.md文件:说明了本项目的主要信息和使用方法。
    \item requirements.txt文件:说明了本项目所需环境中的依赖包。
    \item Dataset\_Group文件夹:存放待分割样本和标签。
    \item train\_data文件夹:存放经切片之后的二维图像,可供深度学习使用。
    \item output文件夹:存放经过算法之后输出的文件,文件夹下共由5个子文件夹,对应5种分割和结果后处理的方法。
    \item run.bat文件:命令行运行脚本,用户若需要再不同的分割方式下进行切换,可以直接在该文件中修改参数实现。
    \item main.py文件:整个项目的主函数,包含了从数据读入,调用算法和结果评估,输出结果的全过程。
    \item iotest.py/sitk\_test.py/iotest.nii.gz文件:项目实现过程中的调试文件和调试输出,用户使用时不要运行。
    \item prepare.py文件:用于将原始数据切片并上采样至256$\times$256的二维图像,其结果输出为jpg格式,存放在train\_data文件夹中。
    \item otsu.py文件:实现了三维的Otsu阈值分割函数。
    \item region\_growing.py文件:实现了三维的区域增长分割函数。
    \item validation.py文件:实现了混淆矩阵(confusion matrix)和所有评价指标的求取。
    \item utils.py文件:存放运行过程中所需的常量。实现其余所有需要用到的辅助函数(如输入输出、可视化、形态学算法等)
    \item result.json文件:存放分割算法的评价指标原始数据。
    \item result\_to\_csv.py文件:将json中的原始数据读出后转换并整理为csv格式。
    \item result.csv文件:存放本项目各方法的最终对比结果。
\end{itemize}

\subsection{预处理方法}

\section{实现方法与结果}
\subsection{基于多阈值Otsu的图像分割}
\subsubsection{传统的多阈值Otsu方法}
\subsubsection{改进后的Otsu方法}


\subsection{基于区域增长的图像分割}
\subsubsection{传统的区域增长方法}
\subsubsection{改进后的区域增长方法}

\subsection{基于深度学习的图像分割}


\section{项目评价}
\subsection{各方法效果比较}
\subsection{项目优势}
\subsection{项目缺点}

\section{成员分工与贡献}
\subsection{裴奕博}
\begin{itemize}
    \item 完成了输入输出、可视化、结果评估、运行脚本等辅助函数的实现。
    \item 尝试了用Pytorch实现UNet等网络结构进行深度学习的算法。
    \item 与组内其他成员共同讨论,提出了分割算法改进的思路。
    \item 完成了后续说明文档的书写,并与组内其他成员共同完成了项目报告。
\end{itemize}
\subsection{丁一}
\subsection{陈波}
\subsection{栗行健}

\section{感想与展望}

\end{document}
