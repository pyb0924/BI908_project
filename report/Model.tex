\documentclass[UTF8]{ctexart}
\usepackage{graphicx}
\usepackage{float}
\usepackage{geometry}
\usepackage{fancyhdr}
\usepackage{lastpage}
\geometry{left=2.54cm,right=2.54cm,top=3.18cm,bottom=3.18cm}%页边距
\pagestyle{fancy}
\lhead{\includegraphics[scale=1]{sjtu-logo-red.pdf}}  
\rhead{BI908 多种方法实现的脑肿瘤分割} 
\cfoot{第 \thepage\ 页\ 共 \pageref{LastPage} 页} 

\begin{document}

\begin{titlepage}
    \begin{center}
        \includegraphics[width=0.8\textwidth]{sjtu-name-blue.pdf}\\[1cm]
        \textsc{\Huge \bfseries 课程报告}\\[1.5cm]
        \includegraphics[width=0.3\textwidth]{sjtu-badge-blue.pdf}\\[0.5cm]    

        \Huge \bfseries{BI908 多种方法实现的脑肿瘤分割}\\[1cm]
        \Large \bfseries{518021910971 裴奕博}\\
        \Large \bfseries{学号 丁一}\\
        \Large \bfseries{学号 陈波}\\
        \Large \bfseries{学号 栗行健}
    \end{center}
\end{titlepage}
\tableofcontents
%正文

\section{项目简介与预处理}
\subsection{项目背景}
\subsection{所选数据}
\subsection{各文件(夹)功能}
\subsection{预处理方法}

\section{实现方法与结果}
\subsection{基于多阈值Otsu的图像分割}
\subsubsection{传统的多阈值Otsu方法}
\subsubsection{改进后的Otsu方法}


\subsection{基于区域增长的图像分割}
\subsubsection{传统的区域增长方法}
\subsubsection{改进后的区域增长方法}

\subsection{基于深度学习的图像分割}


\section{项目评价}
\subsection{各方法效果比较}
\subsection{项目优势}
\subsection{项目缺点}

\section{成员分工与贡献}
\subsection{裴奕博}
\subsection{丁一}
\subsection{陈波}
\subsection{栗行健}

\section{感想与展望}

\end{document}
